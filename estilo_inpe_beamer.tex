\documentclass[10pt]{beamer}

% Tema padrão
\usetheme{default}

% Carregamento dos pacotes utilizados

\usepackage[brazilian]{babel}
%\usepackage[brazilian]{babel} % usar este pacote se a apresentação for em inglês
\usepackage{times}
\usepackage[T1]{fontenc}
\usepackage{lipsum}
\usepackage{graphicx}

% Definição do estilo das fontes
\usefonttheme{professionalfonts} % using non standard fonts for beamer
\usefonttheme{serif} % default family is serif
\usepackage{fontspec}
\setmainfont{Liberation Sans}

% Imagem de fundo dos frames
\usebackgroundtemplate%
{%
	\includegraphics[width=\paperwidth,height=\paperheight]{fundo_slide_inpe.png}%
}

% Remove a barra de navegação dos frames
\beamertemplatenavigationsymbolsempty

% Rodapé dos frames
\makeatother
\setbeamertemplate{footline}
{
	\leavevmode%
	\hbox{%
	\begin{beamercolorbox}[wd=.35\paperwidth,ht=2.25ex,dp=1ex,left]{author in head/foot}%
		\hspace*{10ex}\usebeamerfont{author in head/foot}\insertshortauthor
	\end{beamercolorbox}%
	\begin{beamercolorbox}[wd=.65\paperwidth,ht=2.25ex,dp=1ex,right]{title in head/foot}%
		\usebeamerfont{title in head/foot}\insertshorttitle\hspace*{3em}
		\insertframenumber{} / \inserttotalframenumber\hspace*{3ex}
	\end{beamercolorbox}}%
	\vskip4pt%
}
\makeatletter

% Definição das cores do tema
\definecolor{azulinpe}{RGB}{0, 110, 175}
\definecolor{laranjainpe}{RGB}{248, 133, 31}

\setbeamercolor{title}{fg=azulinpe}
\setbeamercolor{frametitle}{fg=azulinpe, bg=laranjainpe}

\setbeamercolor{palette primary}{fg=azulinpe}
\setbeamercolor{palette secondary}{fg=azulinpe}
\setbeamercolor{palette tertiary}{fg=azulinpe}
\setbeamercolor{palette quaternary}{fg=azulinpe}

\setbeamercolor{structure}{fg=azulinpe} % itemize, enumerate, etc
\setbeamercolor{section in toc}{fg=azulinpe} % TOC sections

% Títulos dos frames
\makeatletter
\defbeamertemplate*{frametitle}{mydefault}[1][left]
{
  \ifbeamercolorempty[bg]{frametitle}{}{\nointerlineskip}%
  \@tempdima=\textwidth%
  \advance\@tempdima by\beamer@leftmargin%
  \advance\@tempdima by\beamer@rightmargin%
  \hspace{0.8cm}
  \includegraphics[scale=0.75]{barra_secao.png}
  \hspace{-0.5cm}
  \pgfsetfillopacity{0}
  \begin{beamercolorbox}[sep=0.3cm,#1,wd=0.64\textwidth]{frametitle}
    \usebeamerfont{frametitle}%
    \vbox{}\vskip-1ex%
    \if@tempswa\else\csname beamer@fte#1\endcsname\fi%
    \strut\pgfsetfillopacity{1}\insertframetitle\strut\par%
    {%
      {\usebeamerfont{framesubtitle}\usebeamercolor[fg]{framesubtitle}\insertframesubtitle\strut\par}%
    }%
    \vskip-1ex%
    \if@tempswa\else\vskip-.3cm\fi% set inside beamercolorbox... evil here...
  \end{beamercolorbox}%
}
\makeatother

% Blocos customizados
\newenvironment<>{problock1}[1]{%
  \begin{actionenv}#2%
      \def\insertblocktitle{#1}%
      \par%
      \mode<presentation>{%
        \setbeamercolor{block title}{fg=laranjainpe, bg=azulinpe}
       \setbeamercolor{block body}{fg=azulinpe, bg=white}
       \setbeamercolor{itemize item}{fg=laranjainpe}
       \setbeamertemplate{itemize item}[triangle]
     }%
      \usebeamertemplate{block begin}}
    {\par\usebeamertemplate{block end}\end{actionenv}}

\newenvironment<>{problock2}[1]{%
  \begin{actionenv}#2%
      \def\insertblocktitle{#1}%
      \par%
      \mode<presentation>{%
        \setbeamercolor{block title}{fg=azulinpe, bg=laranjainpe}
       \setbeamercolor{block body}{fg=laranjainpe, bg=white}
       \setbeamercolor{itemize item}{fg=azulinpe}
       \setbeamertemplate{itemize item}[triangle]
     }%
      \usebeamertemplate{block begin}}
    {\par\usebeamertemplate{block end}\end{actionenv}}

% Informações da capa da apresentação
\title{Nome Muito Comprido da Minha Apresentação Muito Comprida}
\author{Nome Muito Comprido do Autor Muito Comprido}
\institute{Instituto Nacional de Pesquisas Espaciais}
\date{\today}

% A partir daqui inicia-se o documento
\begin{document}

% Capa (NÃO MODIFICAR)
{
\setbeamertemplate{footline}{} 
\begin{frame}
	\vspace{1cm}
	\titlepage
\end{frame}
}
 
% Reinício do contador dos frames 
\addtocounter{framenumber}{-1}
 
% Sumário 
\begin{frame}{Sumário}
	\tableofcontents
\end{frame}


\section{Seção1}
\subsection{Subseção}
\subsection{Subseção}
\subsection{Subseção}
\section{Seção2}
\section{Seção3}
\subsection{Subseção}
\subsection{Subseção}
\subsection{Subseção}

\begin{frame}
\frametitle{Title}
\framesubtitle{Subtitle}
\begin{problock1}{Comando {\tt pause}:}
O Comando {\tt pause} pode ser utilizado antes de tabelas,
blocos, equações, listas, figuras etc. Mais um
exemplo:

\pause

\begin{align}
x & = 1 + 2y + 3z \\
3x - y + 2z & = 0 \\
2x + y & = 2 - z
\end{align}
\end{problock1}
\end{frame}

\begin{frame}
\frametitle{Title}
\framesubtitle{Subtitle}
\begin{block}{Observation 1}
Simmons Hall is composed of metal and concrete.
\end{block}
\begin{exampleblock}{Observation 2}
Simmons Dormitory is composed of brick.
\end{exampleblock}
\begin{alertblock}{Conclusion}
Simmons Hall $\not=$ Simmons Dormitory.
\end{alertblock}

\begin{problock1}
gfgfgh
\end{problock1}	

\begin{problock2}
gfgfgh
\end{problock2}	
\end{frame}

\begin{frame}
\frametitle{Title}
\framesubtitle{Subtitle}
\lipsum[1]
\end{frame}

\begin{frame}
\frametitle{Title}
\framesubtitle{Subtitle}
\lipsum[1]
\end{frame}

\begin{frame}
\frametitle{Title}
\framesubtitle{Subtitle}
\lipsum[1]
\end{frame}

% Frame Final (NÃO MODIFICAR)
\usebackgroundtemplate%
{%
	\includegraphics[width=\paperwidth,height=\paperheight]{fundo_slide_inpe_final.png}%
}

\begingroup
\setbeamertemplate{footline}{}
\begin{frame}

\end{frame}
\endgroup

\end{document}
